\section{Deep Learning Approach}
\subsection{Siamese Network Overview}

The \textbf{Siamese network} architecture consists of two primary components: the \textbf{Encoder} and the \textbf{Classifier}. The \textbf{Encoder} is typically a shallow \textbf{Convolutional Neural Network (CNN)} that is designed to map input data, such as images, into a lower-dimensional latent space. The output of the Encoder is an embedding, which captures key features of the input data while preserving the relationships between similar inputs. These embeddings help in differentiating between inputs based on their semantic similarities. The second component, the Classifier, is usually a \textbf{Support Vector Machine (SVM)}, which takes the embeddings generated by the Encoder and classifies them into two categories: \textbf{positive} (similar) or \textbf{negative} (dissimilar).

\begin{figure}[htbp]
\centering
\resizebox{\textwidth}{!}{
\begin{tikzpicture}
\node [draw, rectangle, text width=1cm] (inp) {
Input
\includegraphics[width=1cm]{/home/someone/me/projects/research/res/assets/ieee_ds_sample_dm.png}
};

\node [draw, ellipse  ,text width=2cm, right=of inp] (in) {Image Normalization};
\node [draw, ellipse  ,text width=2cm, above=of in] (cg) {Convert To Grayscale};
\node [draw=none,  rectangle, above=0 of cg] (Dp) {Data Preprocessing};
\node [draw, ellipse  ,text width=2cm, below=of in] (db) {Data Balancing};

\node [draw, dashed, rectangle, right=0 of inp, fit=(in)(cg)(db)] (dp) {};
\node [draw, rectangle, right=of dp, minimum width=2cm, minimum height=2cm] (sn) {Siamese Network};
\node [draw, circle, right=of sn] (op) {Output};


\draw[->, thick] (inp) -- (dp);
\draw[->] (cg) -- (in);
\draw[->] (in) -- (db);
\draw[->, thick] (dp) -- (sn);
\draw[->, thick] (sn) -- (op);

\end{tikzpicture}
}
\caption{Siamese Network Flowdiagram}
\end{figure}
